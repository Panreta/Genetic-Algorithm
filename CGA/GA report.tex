\documentclass{article}
\usepackage[british]{babel}
\usepackage{amsmath,amssymb,graphicx,stmaryrd,latexsym,xcolor,theorem}

%%%%%%%%%% Start TeXmacs macros
\newcommand{\assign}{:=}
\newcommand{\tmaffiliation}[1]{\\ #1}
\newcommand{\tmem}[1]{{\em #1\/}}
\newcommand{\tmop}[1]{\ensuremath{\operatorname{#1}}}
\newcommand{\tmsamp}[1]{\textsf{#1}}
\newcommand{\tmstrong}[1]{\textbf{#1}}
\newcommand{\tmverbatim}[1]{\text{{\ttfamily{#1}}}}
\newenvironment{proof}{\noindent\textbf{Proof\ }}{\hspace*{\fill}$\Box$\medskip}
\newtheorem{corollary}{Corollary}
\newtheorem{definition}{Definition}
{\theorembodyfont{\rmfamily}\newtheorem{example}{Example}}
\newtheorem{lemma}{Lemma}
{\theorembodyfont{\rmfamily}\newtheorem{note}{Note}}
{\theorembodyfont{\rmfamily}\newtheorem{remark}{Remark}}
\newtheorem{theorem}{Theorem}
%%%%%%%%%% End TeXmacs macros

\begin{document}

\title{Review Convergence of Canonical Genetic Algorithm}

\author{
  Tingyu Zhang
  \tmaffiliation{2023.10.16}
}

\maketitle

\section{Introduction}

The target is to use Canonical genetic algorithm(CGA) to tackle the static
optimization problems

of the type:
\begin{equation}
  \max \{ f (b) \mid b \in \mathbb{B}^l \}
\end{equation}
which $\mathbb{B}^l \assign \{ 0, 1 \}^l, \tmop{and} \tmop{has} \tmop{the}
\tmop{condition} 0 < f (b) < \infty \tmop{for} \tmop{all} b \in \mathbb{B}^l
\tmop{and} f (b) \neq \tmop{const} .$

In optimization theory an algorithm is said to converge to the global optimum
if it {\tmem{{\tmem{generates a sequence of
solutions{\tmstrong{\tmverbatim{{\tmsamp{}}}}}}}
{\tmstrong{\tmverbatim{}}}}}or function values in which the global optimum is
a limit value. Especially,

it is proved by means of homogeneous finite Markov chains that the CGA
neverconverges to the global optimum, but modified versions do.Precisely,
probabilistic convergence of the best solution within a population to the
global optimum under elitist selection, which is the theorem the rest of the
paper will prove.

\section{Basic Tools}

We will introduce CGA and finite Markov Chain as the background:

\subsection{CGA}

A genetic algorithm consists of an n-tuple of binary strings $b_i$ of length
l, where the bits of each string are considered to be the genes of an
individual chromosome and where the n-tuple of individual chromosomes is said
to be a population.

\begin{example}
  For a GA problem, suppose we have a genetic algorithm in which each
  chromosome is a binary string of length 5, and our population size is 3. A
  possible representation of the population could be:
  
  Chromosome 1: 10101
  
  Chromosome 2: 11010
  
  Chromosome 3: 00111
  
  The tuple consisting of these three chromosomes, $(10101, 11010, 00111)$,
  represents our population. $b_i \tmop{represents} a \tmop{certain}
  \tmop{chromosome} \tmop{of} \tmop{population}, l = 5, n = 3, i \in
  \llbracket 1, n \rrbracket$
\end{example}

Following the terminology of organic evolution the operations performed on the
population are called mutation, crossover and selection (differential
reproduction,差异化繁殖). Each individual bi represents a feasible
solution of problem (1) and its ob jective function value $f (b_i)$ is said to
be its fitness which is to be maximized.

\begin{remark}
  Fitness function is a judgement criterion that is positive and increase more
  approach
  
  to the better solution. For example, $\max g (x) = - x^2 + 4 x, x \in [1,
  5], \tmop{the} \tmop{corresponding} \tmop{fitness} \tmop{function}$
  
  can be $f (x) = g (x) + 100.$ And $\min g (x) = x^2, x \in [- 10, 10],
  \tmop{the} \tmop{corresponding} \tmop{fitness} \tmop{function} \tmop{can}
  \tmop{be}$
  
  $f (x) = \frac{1}{x^2 + 0.1}, \tmop{where} 0.1 \tmop{is} \tmop{for}
  \tmop{avoiding} \tmop{the} \tmop{denominator} \tmop{to} \tmop{be} 0.$
\end{remark}

The algorithm is sketched as follows:

choose an initial population\\
determine the fitness of each individual\\
perform selection\\
repeat\\
perform crossover\\
perform mutation\\
determine the fitness of each individual\\
perform selection\\
until some stopping criterion applies\\


{\color[HTML]{AAAAFF}Mutation} operates independently on each individual by
probabilistically perturbing each bit string. The event that the j-th bit of
the i-th individual is flipped is stochastically independent and occurs with
probability $p_m \in (0, 1)$. For example, the probability that string $b =
00000$ transitions to string $b' = 10110$ by mutation is $p_m^k (1 - p_m)^{l -
k}$ with $k = 3$ and $l = 5$. Clearly, $k$ is just the Hamming distance $H (b,
b')$ between strings $b$ and $b'$. Therefore the probability that string $b_i$
resembles string $b_i'$ \ after mutation can be aggregated to
\begin{equation}
  P \{ b_i \rightarrow b_i' \} = p_m^{H (b, b')} (1 - p_m)^{l - H (b, b')} > 0
\end{equation}
{\color[HTML]{AAAAFF}Crossover} operator is applied with some probability $p_c
\in [0, 1]$ in order to construct a bit string from at least two other bit
strings chosen at random. Although many crossover operators have been proposed
a description can be omitted because the choice of a speciac crossover
operator does not effect the subsequent analysis.(Check out)

For {\color[HTML]{AAAAFF}Proportional selection}, the population of the next
generation is determined by n independent random experiments. The probability
that individual bi is selected from tuple $(b_1, \ldots, b_n)$ to be a member
of the next generation at each experiment is given by:
\begin{equation}
  P \{ b_i \tmop{is} \tmop{selected} \} = \frac{f
  (b_i)}{\overset{n}{\underset{j = 1}{\sum}} f (b_j)} > 0
\end{equation}

\subsection{Finite Markov Chains}

A finite Markov chain describes a probabilistic trajectory over a finite state
space $S$ of cardinality $| S | = n$, where the states may be numbered from 1
to n.

The probability $p_{\tmop{ij}} (t)$ of transitioning from state $i \in S
\tmop{to} \tmop{state} j \in S$at step t is called the transition probability
from $i \tmop{to} j \tmop{at} \tmop{step} t$.

If the transition probabilities are independent from t, i.e., $p_{\tmop{ij}}
(t) = p_{\tmop{ij}} (s)$ $\tmop{for} \tmop{all} i, j \in S \tmop{and}
\tmop{for} \tmop{all} s ;$ $t \in \mathbb{N}$, the Markov chain is said to be
homogeneous.

\begin{remark}
  Homogeneity is a really crucial property, since only homogeneous Markov
  Chain
  
  can determain every step by transitioning matrix and initial state in the
  formula $p^t = p^0 P^t$.
\end{remark}

\begin{definition}
  A square matrix $A \in \mathcal{M}_n (K) $is said to be
  
  (a) nonnegative ($A \succcurlyeq 0$), if $a_{\tmop{ij}} \geqslant 0
  \tmop{for} \tmop{all} i, j \in \{ 1, \ldots, n \}$
  
  (b) positive ($A \succ 0$), if $a_{\tmop{ij}} > 0 \tmop{for} \tmop{all} i, j
  \in \{ 1, \ldots, n \}$
  
  A nonnegative matrix $A \in \mathcal{M}_n (K)$ is said to be
  
  (c) primitive, if there exists a $k \in \mathbb{N}$ such that $A ^k $is
  positive,
  
  (d) reducible, if A can be brought into the form (with square matrices C and
  T)\left(\begin{tabular}{c}
    C {\tmstrong{{\tmem{0{\tmstrong{}}}}}}\\
    R T
  \end{tabular}\right) by applying the same permutations to rows and columns,
  
  (e) irreducible, if it is not reducible,
  
  (f ) stochastic, if $\sum_{j = 1}^n a_{\tmop{ij}} = 1 (\tmop{summation}
  \tmop{of} \tmop{the} \tmop{row} \tmop{of} \tmop{matrix})  \prefixforall i
  \in \{ 1, \ldots, n \}$
  
  A stochastic matrix $A \in \mathcal{M}_n (K)$ is said to be(g) stable, if it
  has identical rows,
  
  (h) column al lowable, if it has at least one positive entry in each column.
\end{definition}

Quite a lot definitions are made, here comes what they can do:

\

\begin{lemma}
  Let C,M and S be stochastic matrices, where M is positive and S is column
  allowable. Then the product CMS is positive.
\end{lemma}

By the lamma 5, 2 theorem can be proved:

\begin{theorem}
  Let P be a primitive stochastic matrix. Then $P^k$ converges as $k
  \rightarrow \infty$ to a positive stable stochastic matrix $P^{\infty} = 1'
  p^{\infty}$, where $p^{\infty} = p^0 \underset{k \rightarrow \infty}{\lim
  P^k} = p^0 P^{\infty}$ has nonzero entries andis unique regardless of the
  initial distribution.
  
  proof:
  
  1. Positive stable stochastic get directly from definition.
  
  2. By induction, all element in $P^k$ is smaller than 1. And by
  contradiction all eigen-valun of $P^k$
  
  must be smaller than 1. Since by $| I - P | = 0, P \tmop{has} 1 \tmop{as}
  \tmop{eigen} - \tmop{value},$hence $P^k$has 1 as eigen-value. Hence
  convergence exist and not to 0.
\end{theorem}

\begin{theorem}
  Let P be a reducible stochastic matrix, where $C \in \mathcal{M}_m (K)$ is a
  primitive stochastic matrix and $R, T \neq 0$. Then
  \begin{equation}
    P^{\infty} = \underset{k \rightarrow \infty}{\lim P^k} = \underset{k
    \rightarrow \infty}{\lim} \left(\begin{array}{c}
      C^k \hspace{5em} 0\\
      \sum^{k - 1}_{i = 0} T^i \tmop{RC}^{k - i - 1} T^k
    \end{array}\right) = \left(\begin{array}{c}
      C^k 0\\
      R_{\infty} 0
    \end{array}\right)
  \end{equation}
\end{theorem}

is a stable stochastic matrix with $P^{\infty} = 1' p^{\infty}$, where
$p^{\infty} = p^0 P^{\infty}$ is unique regardless of the initial
distribution, and $p^{\infty} \tmop{satisfies} : p_i^{\infty} > 0 \tmop{for} 1
\leqslant i \leqslant m \tmop{and} p_i^{\infty} = 0 \tmop{for} m < i \leqslant
n$.

\section{Markov Chain Analysis of Genetic Algorithms}

Determine the state space $S = \mathbb{B}^N =\mathbb{B}^{l \ast n},
\tmop{where} l \tmop{is} \tmop{the} \tmop{nunber} \tmop{of} \tmop{genes}
(\tmop{or} \tmop{length} \tmop{of} \tmop{chromosome}), \tmop{and} n
\tmop{denotes} \tmop{the} \tmop{population} \tmop{size},$

$\tmop{Each} \tmop{element} \tmop{of} \tmop{the} \tmop{state} \tmop{space}
\tmop{can} \tmop{be} \tmop{regarded}$as an integer number in binary
representation.

($n = \sum^m_{k = 0} 2^k a_k$)

The projection $\pi_k (i)$ picks up the k-th bit segment of length l from the
binary representation of state i and is used to identify single individuals
from the population.

\begin{example}
  For $b_i = 1111010110, \tmop{choose} l = 3, \pi_2 (i)$=101.
\end{example}

The probabilistic changes of the genes within the population caused by the
genetic operators are captured by the transition matrix P, which can be
decomposed in a natural way into a product of stochastic matrices $P = C M S$,
where C, M and S describe the intermediate transitions caused by crossover,
mutation and selection, respectively.

\begin{theorem}
  The transition matrix of the CGA with mutation probability $p_m \in (0, 1)$,
  crossover probability $p_c \in [0, 1]$ and proportional selection is
  primitive.
\end{theorem}

proved by discussing the structure of C,G,A and lemma 1.

\

The result can be used to form the uniqueness limit state

\begin{corollary}
  The CGA with parameter ranges as in Theorem 3 is an ergodic Markov chain,
  i.e., there exists an unique limit distribution for the states of the chain
  with nonzero probability to be in any state at any time regardless of the
  initial distribution.
\end{corollary}

proved by Theorom 1 and Theorom 3.

\begin{remark}
  Ergodic Markov has another definition: the markov chain which is aperiodic,
  irreducible and finite recurrence.
\end{remark}

From the result, we can apparently see that for such Markov chain the initial
distribution $p ^0 $doesn't affect the limit distribution at all !

The ergodicity property has consequences for the convergence behavior of the
CGA. To avoid confusion, a precise de\k{}nition of the term convergence of a
GA is required:

\begin{definition}
  Let $Z_t = \max \{ f (\pi_k^{(t)} (i) \mid k = 1, \ldots, n) \}$ be a
  sequence of random variables representing the best fitness within a
  population represented by state i at step t. A genetic algorithm converges
  to the global optimum, if and only if
  \begin{equation}
    \underset{t \rightarrow \infty}{\lim} P \{ Z_t = f^{\ast} \} = 1,
  \end{equation}
\end{definition}

where$f^{\ast} = \max \{ f (b) \mid b \in \mathbb{B}^l \} $is is the global
optimum of problem (1)\label{(1)}.

By this definition:

\begin{theorem}
  The CGA with parameter ranges as in Theorem 3 does not converge to the
  global optimum.
  
  \begin{proof}
    set $\max \{ f (\pi_k^{(t)} (i) \mid k = 1, \ldots, n) \} < f^{\ast}
    ,p^t_i = \{ \tmop{probability} \tmop{GA} \tmop{in} \tmop{state} i
    \tmop{at} \tmop{step} t \}$;
    
    \underbrace{$P \{ Z_t \neq f^{\ast} \} \geqslant p^t_i$}$\Leftrightarrow P
    \{ Z_t = f^{\ast} \} \leqslant 1 - p^t_i$. Hence $\underset{t \rightarrow
    \infty}{\lim} P \{ Z_t = f^{\ast} \} \leqslant 1 - p^t_i < 1.$
    
    see remark
  \end{proof}
\end{theorem}

\begin{remark}
  Does ``$P \{ Z_t \neq f^{\ast} \} \geqslant p^t_i$ '' stand for the
  optimization may hard to get?
\end{remark}

Can CGA be changed a little to fulfill the convergence? Actually it can be
done by the theorem of

ergodic Markov chain:

\begin{theorem}
  In an ergodic Markov chain the expected transition time between initial
  state i and any other state j is finite regardless of the states i and j,
  i.e. $E (T_{i \rightarrow j}) < \infty .$
\end{theorem}

\subsection{Little Change to make convergence}

To make the result in conformity with Definition 2:

Change: Enlarging the population by adding {\color[HTML]{AAAAFF}super
individual} which does not take part in the evolutionary process.

Cardinality: From $2^{n \ast l} \rightarrow 2^{(n + 1) \ast l}$,give $l$
bits for super indivitual leftmost and is accessible by

$\pi_0 (i) \tmop{at} \tmop{state} i.$

Require :better the super individual's fitness the higher the position of the
corresponding state in the matrix.

The extended transition matrices are written as diagonal matrices with $2^l
\tmop{square} \tmop{matrices} C, M, S \quad$

of size $2^{\tmop{nl}} \times 2^{\tmop{nl}}$.

The copy operation is represented by an upgrade matrix U which upgrades an
intermediate state containing an individual better than its super individual
to a state where the super individual equals the better individual.

Let $b = \tmop{argmax} \{ f (\pi_k  (i)) \mid k = 1, \ldots, n \} \in
\mathbb{B}^l : =$\{$\tmop{best} \tmop{individual} \tmop{of} \tmop{the}
\tmop{population} \tmop{at} \tmop{any} \tmop{state} i$(exclude super
indivitual)\}.Then $u_{\tmop{ij}} = 1 \tmop{if} f (\pi_0 (i)) < f (b)
\tmop{with} j \overset{\tmop{def}}{=} (b, \pi_1 (i), \pi_2 (i), \ldots, \pi_n
(i)) < S$, otherwise $u_{\tmop{ii}} = 1$.(WHY?)

Thus, there is exactly one entry in each row, which does not hold for the
columns because for every state $j \in s$with $f (\pi_0 (i)) < \max \{ f
(\pi_k  (i)) \mid k = 1, \ldots, n \} $one gets $u_{\tmop{ij}} = 0$ for all $i
\in S.$

\

\

\begin{theorem}
  The canonical GA as in Theorem 3 maintaining the best solution found over
  time after selection converges to the global optimum.
\end{theorem}

\

\

\begin{theorem}
  The canonical GA as in Theorem 3 maintaining the best solution found over
  time before selection converges globally optimal.
\end{theorem}

\begin{note}
  Theorems 6 and 7 do not cover the case of elitist selection.\\
  
\end{note}

\section{Discussion of results with respect to the schema theorem}

A schema(模式) S describes a specific type of subsets of the feasible region
$\mathbb{B}^l$ of problem (1) which is again assumed to have only one global
optimal point $b^{\ast} \in \mathbb{B}^l .$

Usually, these subsets are represented by a string of length l over the
alphabet $\{ 0, 1, \# \} .$

The utility of schema S restricted to multiset X is defined as the average
objective function value over all elements contained in $S \cap X$:
\begin{equation}
  u (S, X) \assign \frac{1}{| S \cap X |} \underset{b \in S \cap X}{\sum} f
  (b)
\end{equation}
Then schema theorem is
\begin{equation}
  E [| S \cap X |] \geqslant | S \cap X_t | \frac{u (S, X_t)}{u (\mathcal{S},
  X_t)} (1 - c (S, X_t)) (1 - m (S, X_t))
\end{equation}
almost surely, where $(X_t)$ is the sequence of populations generated by the
CGA and c(.) and m(.) are bounds for the probability that an element of subset
S is modfied by crossover and mutation respectively, so that the resulting
element is not contained in subset S.

\begin{remark}
  (7) state that the number of individuals in population $X_{t + 1} $with
  above average fitness is expected to be no smaller than in population $X_t
  $,if the probabilities c(.) and m(.) are sufficiently small. But not fairly
  demonstrate the convergence to the global optimum.
  
  Generally, the specific content of the schema theorem is: under the action
  of genetic operators selection, crossover, and mutation, schemas with low
  order, short defining length, and average fitness higher than the population
  average fitness will grow exponentially in the offspring.(Without proof)
\end{remark}

Technically, it is neccessary and sufficient that $\underset{t \rightarrow
\infty}{\lim} E [I_t] = 1$ which implies $\underset{t \rightarrow
\infty}{\lim} E \left[ \{ b^{\ast} \} \bigcap X_t \right] \geqslant 1,$

where
\begin{equation}
  I_t \assign h (b^{\ast}, X_t) \assign \left\{ \begin{array}{l}
    1, \tmop{if} b^{\ast} \in \{ \pi_1 (X_t), \ldots, \pi_n (X_t) \}\\
    0, \tmop{otherwise}
  \end{array} \right.
\end{equation}
means when $t > N$ for certain $N \in \mathbb{N}, b^{\ast} $will be in one of
the states.

In particular:

\begin{lemma}
  \
  
  $(a)  \underset{t \rightarrow \infty}{\lim} E [I_t] = 1 \Leftrightarrow
  \underset{t \rightarrow \infty}{\lim} P \{ Z_t = f^{\ast} \} = 1$
  
  $(b)  \underset{t \rightarrow \infty}{\lim} E [I_t] = 1 \Rightarrow
  \underset{t \rightarrow \infty}{\lim} E \left[ \{ b^{\ast} \} \bigcap X_t
  \right] \geqslant 1$
\end{lemma}

Proof:

$(a)$ Take $I_t = 1_{\{ \pi_1 (X_t), \ldots, \pi_n (X_t) \}} (b^{\ast}),
\tmop{and} b^{\ast} \in \{ \pi_1 (X_t), \ldots, \pi_n (X_t) \} \Leftrightarrow
\{ I_t = 1 \} \Leftrightarrow \{ Z_t = f^{\ast} \}$

$(b)$ $g (b^{\ast}, X_t) \assign \tmop{count} \tmop{the} \tmop{number}
\tmop{of} \tmop{optimal} \tmop{solution} b^{\ast} \tmop{in} \tmop{population}
X_{t.}$ Compare to $h (b^{\ast}, X_t)$ which only count once $\Rightarrow$ $g
(b^{\ast}, X_t) \geqslant h (b^{\ast}, X_t) \Rightarrow$
\begin{equation}
  \underset{t \rightarrow \infty}{\lim} E \left[ \{ b^{\ast} \} \bigcap X_t
  \right] = \sum_{i = 1}^{| S |} g (b^{\ast}, X_t) \cdot p_i^{\infty}
  \geqslant \sum_{i = 1}^{| S |} h (b^{\ast}, X_t) \cdot p_i^{\infty} =
  \underset{t \rightarrow \infty}{\lim} E [I_t] = 1
\end{equation}


Converse for $(b)$ is not true:

\

\begin{remark}
  For a CGA there is a minimal probability bounded from zero to lose the
  global optimum solution at each generation. It follows from the
  Borel-Cantelli Lemma that this event will occur with probability one.
  
  Borel-Cantelli Lemma: $\{ E_n \} \tmop{belongs} \tmop{to} a \tmop{certain}
  \tmop{probability} \tmop{space}, \tmop{if} \sum_{n = 1}^{\infty} \mathbb{P}
  (E_n) < \infty \Rightarrow \mathbb{P} \left( \underset{n \rightarrow
  \infty}{\lim} \sup (E_n) \right) = 0$
  
  Intuition : For converge series like $\{ a_n \}$, $\sum_{n = N}^{\infty} a_n
  \rightarrow 0$ as $N  \rightarrow \infty$.
\end{remark}

conclude,the global solution will be lost and found infinitely often
$\rightarrow$the sequence $\left( \left| \left\{ b^{\ast} \bigcap X_t \right\}
\right| \right)$ is an irreducible markov chain on the state space $\{ 0,
\ldots, n \}$ $\rightarrow$ does not converge to the best solution although
the expectation does

\subsection{Analysis the bounds for the probabilities of losing and generating
the optimal solution}

Assume that $k \geqslant 1$ optimal solutions are contained in population Xt
at generation t:

The crossover operator may destroy or assemble some optimal solutions, so that
there are k optimal solutions within the population after crossover.

\subsubsection{The Optimal Solution Gets Lost}

$1. k \geqslant 1.$

The probability that at least one bit of an optimal solution is ipped is
given by:

$p_F \assign 1 - \underset{\tmop{all} l - \tmop{bit} \tmop{do} \tmop{not}
\tmop{change}}{\underbrace{(1 - p_m)^l}} > 0,$the probability that all k
optimal solutions are destroyed becomes:

$p_F^k (1 - p_F)^{n - k} > 0, \tmop{bounded} \tmop{below} : \gamma_1 \assign
\min \left\{ \underset{p_F < 1 - p_F}{\underbrace{p_F^n}}, \underset{1 - p_F <
p_F}{\underbrace{(1 - p_F )^{n - 1}}} \right\} > 0$

\

$2. k = 0. (\tmop{all} \tmop{optimal} \tmop{solutions} \tmop{have}
\tmop{been} \tmop{destroyed} \tmop{by} \tmop{crossover})$ \

The probability that all bits within the population remain unaltered:

$(1 - p_m)^{n \cdot l} = (1 - p_F )^n \assign \gamma_2 > 0.$

$\tmop{the} \tmop{probability} \tmop{that} \tmop{the} \tmop{optimal}
\tmop{solution} \tmop{is} \tmop{lost} \tmop{after} \tmop{crossover} \tmop{and}
\tmop{mutation} \tmop{is} \tmop{at} \tmop{least} :$

$p_L = \min (\gamma_1, \gamma_2) = \min \{ p_F^n, (1 - p_F )^{n - 1}, (1 - p_F
)^n \} = \min \{ p_F^n, (1 - p_F )^n \} > 0 (\tmop{all} n \tmop{destoyed},
\tmop{at} \tmop{least} \tmop{one} \tmop{not}, \tmop{all} \tmop{not})$

\subsubsection{The Optimal Solution Genrating}

It remains to derive the bound for the probability to generate an optimal
solution:

The probability that mutation generates the optimal solution ${b^{\ast}}^{}$
from individual $b_i :$

$p_{B_i} \assign p_m^{H (b_i, b^{\ast})} (1 - p_m)^{l - H (b_i, b^{\ast})} > 0
;$ Bounded below by:$p_B \assign \min \{ p_{B_i} \mid i \in \llbracket 1, n
\rrbracket \} > 0.$

The probability that this event occurs at least once is:$p_G \assign 1 - (1 -
p_B)^n > 0.$

\

Next, consider the selection operator:

Assume that only one optimal solution has been generated by mutation with
probability $p_G .$

The probability to select the optimal solution is given by:\quad$p_{b^{\ast}}
\assign \frac{f (b^{\ast})}{\sum_{j = 1}^n f (b_j)} > 0$,

The probability that this event occurs at least once becomes: $p_S \assign 1 -
(1 - p_{b^{\ast}} )^n > 0.$

The probability that a global solution is generated by mutation and that it
survives the selection procedure can be bounded:$p_G \cdot p_S > 0.$



\

\

\

\

\

\resizebox{700}{345}{\includegraphics{file:///C:/Users/Pluviophile/Documents/Tencent
Files/2170139052/FileRecv/MobileFile/note.jpg}}



\

\

\

\

\

\end{document}
